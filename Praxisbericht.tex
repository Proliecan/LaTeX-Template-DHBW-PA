% ------------------------------------------------------------
% LaTeX Template für die DHBW zum Schnellstart!
% ------------------------------------------------------------
% ---- Präambel mit Angaben zum Dokument
\input{Inhalt/00_Latex/praeambel}

% ---- Elektronische Version oder Gedruckte Version?
% ---- Unterschied: Die elektronische Version enthält keinen Platzhalter für die Unterschrift
\usepackage{ifthen}
\newboolean{e-Abgabe}
\setboolean{e-Abgabe}{false}    % false=gedruckte Fassung

% ---- Persönlichen Daten:
\newcommand{\titel}{Template \LaTeX}
\newcommand{\titelheader}{Titel welcher im Header auftaucht}
\newcommand{\arbeit}{Projektarbeit 1 (T3\_2000)}
\newcommand{\studiengang}{Informatik}
\newcommand{\studienjahr}{2015}
\newcommand{\autor}{Vorname Nachname}
\newcommand{\autorReverse}{Nachname, Vorname}
\newcommand{\verfassungsort}{Karlsruhe}
\newcommand{\matrikelnr}{0000000}
\newcommand{\kurs}{TINF15B1}
\newcommand{\bearbeitungsmonat}{Januar 2018}
\newcommand{\abgabe}{01. Februar 2018}
\newcommand{\bearbeitungszeitraum}{01.10.2017 - 31.01.2018}
\newcommand{\firmaName}{FIRMENNAHME}
\newcommand{\firmaStrasse}{FIRMENSTRASSE}
\newcommand{\firmaPlz}{FIRMENORT}
\newcommand{\betreuerFirma}{B-Vorname B-Nachname}
\newcommand{\betreuerDhbw}{DH-Vorname DH-Nachname}

\input{Inhalt/00_Latex/kopfundFusszeile}

% ---- Hilfreiches
\newcommand{\zB}{z.\,B. }   % "z.B." mit kleinem Leeraum dazwischen (ohne wäre nicht korrekt)
\newcommand{\dash}{d.\,h. }

\newcommand{\code}[1]{\texttt{#1}} % Ist einfacher zu schreiben als ständig \texttt und erlaubt
                                   % Änderungen im Nachhinein, wenn man z.B. Inline-Code anders stylen möchte.

% ---- Silbentrennung (falls LaTeX defaults falsch / nicht gewünscht sind)
\hyphenation{HANA}         % anstatt HA-NA
\hyphenation{Graph-Script} % anstatt GraphS-cript

% ---- Beginn des Dokuments
\begin{document}
\setlength{\parindent}{0pt}              % Keine Paragraphen Einrückung.
                                         % Dafür haben wir den Abstand zwischen den Paragraphen.
\setcounter{secnumdepth}{2}              % Nummerierungstiefe fürs Inhaltsverzeichnis
\setcounter{tocdepth}{1}                 % Tiefe des Inhaltsverzeichnisses. Ggf. so anpassen,
                                         % dass das Verzeichnis auf eine Seite passt.
\sffamily                                % Serifenlose Schrift verwenden.

% ---- Vorspann
% ------ Titelseite
\singlespacing
\thispagestyle{empty}
\begin{titlepage}
\enlargethispage{4cm}

\begin{figure}           % Logo vom Ausbildungsbetrieb und der DHBW
	% \vspace*{-5mm} % Sollte dein Titel zu lang werden, kannst du mit diesem "Hack" 
	%                  den Inhalt der Seite nach oben schieben.
	\begin{minipage}{0.49\textwidth}
		\flushleft
		\missingfigure[figwidth = 8cm, figheight = 2.5cm]{FIRMENLOGO}
	\end{minipage}
	\hfill
	\begin{minipage}{0.49\textwidth}
		\flushright
		\includegraphics[height=2.5cm]{Bilder/Logos/Logo_DHBW.pdf} 
	\end{minipage}
\end{figure} 
\vspace*{0.1cm}

\begin{center}
	\huge{\textbf{\titel}}\\[1.5cm]
	\Large{\textbf{\arbeit}}\\[0.5cm]
	\normalsize{im Rahmen der Prüfung zum\\[1ex] \textbf{Bachelor of Science (B.Sc.)}}\\[0.5cm]
	\Large{des Studienganges \studiengang}\\[1ex]
	\normalsize{an der Dualen Hochschule Baden-Württemberg Karlsruhe}\\[1cm]
	\normalsize{von}\\[1ex] \Large{\textbf{\autor}} \\[1cm]
	% Hinweis: Manche Dozenten möchten einen Hinweis auf den Sperrvermerk auf der Titelseite.
	% \large{{\color{red}- Sperrvermerk -}}\\[1cm]
\end{center}

\begin{center}
	\vfill
	\begin{tabular}{ll}
		Abgabedatum:                     & \abgabe \\[0.2cm]
		Bearbeitungszeitraum:            & \bearbeitungszeitraum \\[0.2cm]
		Matrikelnummer, Kurs:            & \matrikelnr , \kurs \\[0.2cm]
		Ausbildungsfirma:                & \firmaName \\
		                                 & \firmaStrasse \\
		                                 & \firmaPlz \\[0.2cm]
		Betreuer der Ausbildungsfirma:   & \betreuerFirma \\[0.2cm]
		Gutachter der Dualen Hochschule: & \betreuerDhbw \\[2cm]
	\end{tabular} 
\end{center}
\end{titlepage}
  % Titelseite
\newcounter{savepage}
\pagenumbering{Roman}                    % Römische Seitenzahlen
\onehalfspacing

% ------ Erklärung, Sperrvermerk, Abstact
\include{Inhalt/01_Standard/erklaerung}
\chapter*{Sperrvermerk}
Die nachfolgende Arbeit enthält vertrauliche Daten der:
\begin{quote}
	\firmaName \\
	\firmaStrasse \\
	\firmaPlz
\end{quote}

\vspace{0.5cm}

Der Inhalt dieser Arbeit darf weder als Ganzes noch in Auszügen Personen außerhalb des Prüfungsprozesses und des Evaluationsverfahrens zugänglich gemacht werden, sofern keine anderslautende Genehmigung vom Dualen Partner vorliegt.

\include{Inhalt/02_Abstract/abstract-en}
\include{Inhalt/02_Abstract/abstract-de}

% ------ Inhaltsverzeichnis
\singlespacing
\tableofcontents

% ------ Verzeichnisse
\renewcommand*{\chapterpagestyle}{plain}
\pagestyle{plain}
\include{Inhalt/03_Verzeichnisse/formelgroessen}
\include{Inhalt/03_Verzeichnisse/abkuerzungen}
\listoffigures                          % Erzeugen des Abbildungsverzeichnisses 
\listoftables                           % Erzeugen des Tabellenverzeichnisses
\renewcommand{\lstlistlistingname}{Quellcodeverzeichnis}
\lstlistoflistings                      % Erzeugen des Listenverzeichnisses
\setcounter{savepage}{\value{page}}


% ---- Inhalt der Arbeit
\cleardoublepage
\pagenumbering{arabic}                  % Arabische Seitenzahlen für den Hauptteil
\setlength{\parskip}{0.5\baselineskip}  % Abstand zwischen Absätzen
\rmfamily
\renewcommand*{\chapterpagestyle}{scrheadings}
\pagestyle{scrheadings}
\onehalfspacing
\include{Inhalt/04_Inhalt/einleitung}
\include{Inhalt/04_Inhalt/formatText}
\include{Inhalt/04_Inhalt/abbildungen}
\include{Inhalt/04_Inhalt/mathematische-formeln}
\include{Inhalt/04_Inhalt/quellcode}
\include{Inhalt/04_Inhalt/literaturHinweis}

% ---- Literaturverzeichnis
\cleardoublepage
\renewcommand*{\chapterpagestyle}{plain}
\pagestyle{plain}
\pagenumbering{Roman}                   % Römische Seitenzahlen
\setcounter{page}{\numexpr\value{savepage}+1}
\printbibliography[title=Literaturverzeichnis]

% ---- Anhang
\appendix
%\clearpage
%\pagenumbering{Roman}  % römische Seitenzahlen für Anhang

\newpage
\end{document}
